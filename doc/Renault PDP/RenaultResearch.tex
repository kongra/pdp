\section{Renault}
Renault publishes their data on the web today. This is done via an API on the web. 
They have decided to represent their data as a graph which their API helps third
party application traverse. \cite{SemWebAppRes} This is because of vast amount 
specifications a car may contain, by Renault's calculations 
they have around $10^{20}$ different combinations of cars with different specifications.\cite{ren1} 
This is why they have chosen to represent their data as an API. The API is a way to traverse the RDF graph 
of possibilities and find a smaller result set of cars. This also allows Renault 
to handle the reasoning on their side, which can be a big operation with $10^{20}$ 
of combinations. Reasoning on their side would also relieve a lot of performance 
pressure from the applications wanting to use this data. This decision is built 
upon observations that the environments which might want to use their data is 
still young and does not got the reasoning capabilities which would be expected.

This is done with Renault's Configuration Ontology. It is a way to traverse the 
RDF graph of possibilities to find the range of cars they manufacturers without 
the need for heavy reasoning. They call it traversal of "Configurations`` to end up with 
"Partially Defined Products"(PDP)\footnote{Partially Defined Product is a way 
of defining a product without using all of the features. For instance a car which 
may have many features, but we describes it as a car with mp3 and sunroof.}.
Their thoughts of how it is meant to be used is for a user to chose configurations 
on the fly. This means that for each selection the user gets one step closer to a 
fully defined product. 

All possible features has a finite set of options for instance fuel type. There 
the user can chose from gasoline, diesel, electric and gasoline-electric hybrid.\cite{confOnt}
With this it is possible to create a GUI to chose features wanted in your car. 
Renault are also working on allowing to query their API with several specifications like this.
\[ configService?chosenSpec=spec1\&chosenSpec=spec2\&... \]
Here if two features are chosen, this query will return the next list of 
possible features following that the first two are chosen.


\subsection{Data representation}
Here we will take a look at how Renault have chosen to represent their data about
car models and their specifications. Since there is a underlying graph through all the
configuration possibilities, there is also a default starting point.
\begin{verbatim}
http://uk.co.rplug.renault.com/docs#this
\end{verbatim}
At the default starting point one will be presented with all the car models, like Renault Twingo, Renault
have present in their API. There are also links to each individual car model's lexicon. 
In this lexicon one can chose different specifications which may lead to a valid configuration link.
This means that at this point one have chosen a valid specification to a specific model.
\subsubsection{Resource values}
To chose a specification one will be presented with some options of values. 
For instance that one can chose which kind of gearbox, how much CO2 emission
and many more other specifications. 
Since there are no way of determine which specification is representing for instance fuel type 
with the Configuration Ontology, Renault has solved this by adding a configuration id
which represent what kind of specifications it represent. In the example below there
is the configuration variable for fuel type.
\tiny\begin{verbatim}
:var_PT1628
  a       owl:Class , co:ConfigurationVariable ;
  rdfs:label "Fuel Type"@en ;
  co:confVarId "PT1628" ;
  co:hasValue <http://uk.co.rplug.renault.com/spec/BAm/PT1628_diesel#this> , 
	      <http://uk.co.rplug.renault.com/spec/BAm/PT1628_unleaded_petrol#this> ;
  co:lexicon :this ;
  owl:oneOf (<http://uk.co.rplug.renault.com/spec/BAm/PT1628_diesel#this> 
	      <http://uk.co.rplug.renault.com/spec/BAm/PT1628_unleaded_petrol#this>) .
\end{verbatim}
\normalsize
Further one have to match the value from the specification with what the user may have chosen.
This is done within the Specification resource which holds the particular value. Here the actual
specification value is stored in another label as seen in the example below.
\tiny\begin{verbatim}
<http://uk.co.rplug.renault.com/spec/BAm/PT1628_diesel#this>
      a       co:Specification , :var_PT1628 ;
      rdfs:label "Diesel"@en ;
      co:specId "PT1628_diesel" .
\end{verbatim}
\normalsize
If we follow this specification's URI we will be able to chose the configuration link and then start 
the process of choosing another specification over again, but this time with a smaller set of options
since fuel type is already chosen.

The both resources show above there are an id contained. These two id properties are
not mention in the Configuration Ontology reference.\cite{confOnt}
I assume that these ids are made for Renault to keep track of specifications and Configuration Variables.
In the ontology's reference it is mentioned that it is possible to represent each variable in different
ways than with their classes. One example they show is with the use of the Vehicle Sales Ontology.

In figure \vref{valueChart} one can see a simplified figure of the representation of data within the Renault data graph.
This figure does not show the constraints between specification, but they are held within a co:Configuration. 
As described earlier a Configuration is responsible for keeping track of which Specifications that are allowed
at this point of the configuration process.

\begin{figure}
  \centering
      \includegraphics[scale=0.4]{RenaultValueChart.png}
  \caption{RenaultValueChart}
\end{figure}\label{valueChart}

